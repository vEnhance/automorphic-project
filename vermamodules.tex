\input{preamble}

% OK, start here.
%
\begin{document}

\title{Verma modules and the category $\mathcal O$.}


\maketitle

\phantomsection
\label{section-phantom}


\tableofcontents

A good reference for the category $\mathcal O$ is the book \cite{Humphreys-O} of Humphreys.


In this section, all Lie algebras and representations are over an algebraically closed field in characteristic zero, which for notational simplicity we take to be $\mathbb C$.


\section{Verma modules}
\label{section-Verma-modules}
We have seen \ref{liestructure-theorem-complete-reducibility} that finite-dimensional representations of semisimple Lie algebras are completely reducible. We now want to \emph{construct} those irreducible representations (in particular, to show that there is a unique one up to unique isomorphism for each given weight), and to compute their \emph{characters}.

In specific cases one can do that ``by hand'', constructing first the irreducible representations attached to fundamental weights, and then the rest by taking tensor products of those, and removing copies of the representations already constructed. For instance, for $\mathfrak{sl}_n$ the $n-1$ fundamental representations are the first $n-1$ exterior powers of the standard, $n$-dimensional representation.

For a more systematic approach, it is better to move outside the realm of finite-dimensional representations. The category to consider is motivated by the following definition and lemma:
\begin{definition}
 \label{definition-highest-weight}
 Let $\mathfrak g$ be a semisimple (or reductive) Lie algebra, $\mathfrak b$ a Borel subalgebra, and $\mathfrak h$ its quotient by its commutator. Let $V$ be a representation of $\mathfrak g$. A {\it heighest weight vector} (for the given choice of Borel subgroup) is an eigenvector for $\mathfrak b$, and the eigencharacter $\lambda\in\mathfrak h^*$ is called the {\it weight} of the highest weight vector.
\end{definition}

\begin{lemma}
 \label{lemma-generated-highestweight}
A finite-dimensional representation of a semisimple Lie algebra is generated by its highest weight vectors.
\end{lemma}

\begin{proof}
Since the representation is semisimple by Theorem \ref{liestructure-theorem-complete-reducibility}, it is enough to show that any irreducible representation $V$ contains a highest weight vector. This follows from Lie's theorem \ref{liestructure-theorem-Lie}. 
\end{proof}



Thus, we will attempt to construct all finite-dimensional representations by constructing the universal objects with highest weight.

More precisely, we consider the category of $\mathfrak g$-modules of arbitrary, possibly infinite, dimension (no topology), and for $\lambda\in \mathfrak h^*$ (where $\mathfrak h$ denotes a universal Cartan, later to be identified with a Cartan subgroup of $\mathfrak g$) we let $M_\lambda$ denote the module defined by the following universal property:

\begin{definition}
\label{definition-Verma-module}
Let $\mathfrak g$ be a semisimple Lie algebra, $\mathfrak b$ a Borel subalgebra, and $\mathfrak h$ its reductive quotient.
Fix $\lambda\in \mathfrak h^*$. The {\it Verma module} of highest weight $\lambda$ is a $\mathfrak g$-module $M_\lambda$ with the property 
$$\Hom_{\mathfrak g}(M_\lambda, V) = \Hom_{\mathfrak b} (\mathbb C_\lambda,V),$$
where $\mathbb C_\lambda$ is the one-dimensional module where $\mathfrak b$ acts by the character $\lambda$.
\end{definition}

\begin{lemma}
\label{lemma-Verma-exists}
For every $\lambda\in\mathfrak h^*$, the Verma module $M_\lambda$ exists and can be identified as
$$M_\lambda = U(\mathfrak g)\otimes_{U(\mathfrak b)} \mathbb C_\lambda.$$
\end{lemma}

\begin{proof}
 Simply the universal property of tensor products.
\end{proof}

Let us now fix an opposite Borel $\mathfrak b^-$, identifying $\mathfrak h$ with $\mathfrak b \cap \mathfrak b^-$. We denote by $\mathfrak n$, $\mathfrak n^-$ the nilpotent radicals of $\mathfrak b$, $\mathfrak b^-$.
Notice that, by the PBW theorem, as a $\mathfrak b^-$-module:
\begin{equation}\label{Vermastructure}M_\lambda = U(\mathfrak n^-)\otimes_{\mathbb C} \mathbb C_{\lambda},
\end{equation}
where $U(\mathfrak n^-)$ acts by left multiplication on the first factor, and the $\mathfrak h$-action is the tensor product of the adjoint representation and the representation on $\mathbb C_{\lambda}$. 

Therefore:
\begin{lemma}
\label{lemma-properties-Verma}
 \begin{enumerate}
  \item $M_\lambda$ is $\mathfrak h$-locally finite and semisimple. The ($\mathfrak h$-)weights of $M_\lambda$ are of the form $\lambda - \sum_i c_i\alpha_i$, where $\alpha_i$ range over simple positive roots (we will denote their set by $\Delta$) and $c_i\in \mathbb Z_{\ge 0}$. The weight spaces are finite-dimensional, and the $\lambda$-weight space $M_\lambda^\lambda$ is one-dimensional.
 \item $M_\lambda$ is $\mathfrak n$-locally finite.
 \end{enumerate}
\end{lemma}

\begin{proof}
 The first statement follows immediately from the presentation (\ref{Vermastructure}), and the second from the first and the fact that the action of $\mathfrak n$ raises weights.
\end{proof}

\begin{proposition}
\label{proposition-uniquequotient}
 $M_\lambda$ has a unique irreducible quotient $L_\lambda$. 
\end{proposition}

\begin{proof}
Any proper submodule is spanned by its $\mathfrak h$-eigenspaces (since the $\mathfrak h$-action is locally finite), and 
 if it is proper, it cannot meet $M_\lambda^\lambda$, since this generates $M_\lambda$. Therefore, the sum of all proper submodules is proper.
\end{proof}




\section{The category $\mathcal O$.}
\label{section-category-O}

\begin{definition}
 \label{definition-category-O}
The {\it category $\mathcal O$} is the full subcategory of the category of $\mathfrak g$-modules consisting of those objects which are:
\begin{itemize}
 \item $\mathfrak h$-locally finite and semisimple;
 \item $\mathfrak n$-locally finite;
 \item finitely generated.
\end{itemize}
\end{definition}

By Lemma \ref{lemma-properties-Verma}, Verma modules belong to the category $\mathcal O$. 


\begin{lemma}
\label{lemma-O-properties}
A $\mathfrak g$-submodule or a quotient module of a module in $\mathcal O$ is in $\mathcal O$. The category $\mathcal O$ is a Noetherian abelian category.
\end{lemma}

\begin{proof}
For a $\mathfrak g$-submodule $N\subset M$, if $M$ is  $\mathfrak h$-locally finite and semisimple and $\mathfrak n$-locally finite, so is $N$. Moreover, the universal enveloping algebra $U(\mathfrak g)$ is Noetherian, by Proposition \ref{liegroups-general-proposition-Ug-Noetherian}, hence if $M$ is finitely generated, so is $N$. The same properties for quotient modules are obvious.

The category being Noetherian means that the union of an increasing chain of submodules has to stabilize. But the union is a submodule, hence finitely generated, therefore stabilizes.

The category of $\mathfrak g$-modules is abelian, and it is clear that coproducts (direct sums) of objects in $\mathcal O$ are also in $\mathcal O$. Since submodules and quotient modules (hence, kernels and cokernels) are also in $\mathcal O$, the category is abelian.
\end{proof}

We will eventually see that it is also Artinian, i.e.\ every object is of finite length.

\begin{lemma}
\label{lemma-filtration-Verma}
Every object in $\mathcal O$ has a filtration whose quotients are surjective images of Verma modules.
\end{lemma}

\begin{proof}
 Let $V$ be in $\mathcal O$, and let $W\subset V$ be a finite-dimensional, generating subspace. Without loss of generality, $W$ is $\mathfrak b$-stable (for $U(\mathfrak b)W$ is, in any case, finite-dimensional). By Lie's theorem \ref{liestructure-theorem-Lie}, it has a filtration with one-dimensional quotients. Therefore, $V$ has a filtration with quotients generated by $\mathfrak b$-eigenvectors. Each such representation is the surjective image of a Verma module.
\end{proof}




\begin{definition}
 \label{definition-character}
The {\it character} $\text{ch}_V$ of an object $V$ in the category $\mathcal O$ (or a sub-$\mathfrak h$-module) is the formal sum 
$$\sum_{\lambda \in \mathfrak h^*} \dim V_\lambda \cdot e^\lambda.$$ 
\end{definition}

\begin{lemma}
 \label{lemma-character-inR}
For every $V\in \mathcal O$, the character $\text{ch}_V$ belongs to the ring $C$ of formal sums $\sum_{\lambda\in \mathfrak h^*} c(\lambda) e^\lambda$, where $c_\bullet: \mathfrak h^*\to \mathbb Z$ is supported in a finite number of translates of the negative root monoid, and multiplication defined by $e^\lambda \cdot e^\mu = e^{\lambda+\mu}$.

The character map $\text{ch}: \mathcal O \to C$ factors through the Grothendieck group $\mathbb Z[\mathcal O]$ of the category $\mathcal O$; moreover, for any finite-dimensional $\mathfrak g$-module $L$, and any $M\in\mathcal O$, we have $L\otimes M\in \mathcal O$, and $\text{ch}_L \cdot \text{ch}_M = \text{ch}_{L\otimes M}$. 
\end{lemma}

Recall that the \emph{Grothendieck group} of an abelian category $\mathcal C$ is the free group on its objects, modulo the relation: $[B]=[A]+[C]$ for every short exact sequence $0\to A \to B\to C\to 0$. We will eventually see that the Grothendieck group of $\mathcal O$ is generated by Verma modules, in fact: it is free on the set of Verma modules.

\begin{proof}
 It is clear from the definition that the character factors through the Grothendieck group. It follows from Lemma \ref{lemma-filtration-Verma} that the support of the character of any object is contained in the support of the character of a finite direct sum of Verma modules. By Lemma \ref{lemma-properties-Verma}, the support of the characters of those are translates of the negative root monoid.
\end{proof}


\section{The case of $\mathfrak{sl}_2$, and application to general Lie algebras.}
\label{section-sl2}

Let $\mathfrak g = \mathfrak{sl}_2$. We identify $\mathfrak h^* \simeq \mathbb C$, by applying the positive root $\check\alpha$. Under this, the half-sum of positive roots $\rho = \frac{\alpha}{2}$ corresponds to $1$.

\begin{lemma}
\label{lemma-Verma-sl2}
For $\mathfrak g = \mathfrak{sl}_2$, $M_{\lambda-\rho}$ is irreducible, unless $\lambda \in \mathbb Z_{>0}$, in which case there is an exact sequence:
$$ 0 \to M_{-\lambda-\rho}\to M_{\lambda-\rho} \to L_{\lambda-\rho} \to 0.$$
\end{lemma}

\begin{proof}
 Every submodule must have a highest weight vector, which must be of the form $F^nv_{\lambda-\rho}$. We compute that:
$$ EF^n v_{\lambda-\rho} = n (\lambda-n) F^{n-1} v_{\lambda-\rho},$$
therefore for it to be zero, for some $n>0$, we must have $\lambda\in \mathbb Z_{>0}$.
\end{proof}

We return to the case of a general semisimple $\mathfrak g$. Then:

\begin{proposition}
\label{proposition-sa-quotient}
 If $\alpha$ is a simple root such that $\left< \lambda,\check\alpha\right>\in \mathbb Z_{>0}$ then there is an embedding: $M_{w_\alpha\lambda-\rho}\hookrightarrow M_{\lambda-\rho}$. The quotient $V=M_{\lambda-\rho}/M_{w_\alpha\lambda-\rho}$ has the property that it is locally $(\mathfrak{sl}_2)_\alpha$-finite, where $(\mathfrak{sl}_2)_\alpha$ denotes the embedding of $\mathfrak{sl}_2$ into $\mathfrak g$ determined by the root $\alpha$. The character of any subquotient of $V$ is $w_\alpha$-stable.
\end{proposition}

\begin{proof}
 As in Lemma \ref{lemma-Verma-sl2}, we calculate that there is a highest weight vector with weight $w_\alpha\lambda$, hence there is a non-trivial map: $M_{w_\alpha\lambda-\rho}\to M_{\lambda-\rho}$. Since $M_{w_\alpha\lambda-\rho}, M_{\lambda-\rho} \simeq U(\mathfrak n^-)$ as $U(\mathfrak n^-)$-modules, and $U(\mathfrak n^-)$ does not have zero divisors, such a map has to be injective.

 With notation $(H_\alpha,E_\alpha,F_\alpha)$ for the $\mathfrak{sl}_2$-triple corresponding to $\alpha$, we need to show that the quotient is $F_\alpha$-locally finite. (Finiteness under the other two is automatic for the category $\mathcal O$.) If $V'$ is the set of $F_\alpha$-finite vectors, then $V'\ni v_{\lambda-\rho}$; we claim that $V'$ is $\mathfrak g$-stable. Indeed, we have a homomorphism of $F_\alpha$-modules: $\mathfrak g\otimes V'\to V$, where $F_\alpha$ acts on $\mathfrak g$ via the adjoint representation. But $\mathfrak g$ is $F_\alpha$-finite and $V'$ is $F_\alpha$-locally finite, hence their tensor product is locally finite, therefore $\mathfrak gV'\subset V'$. Together with $v_{\lambda-\rho}\in V'$, this implies that $V'=V$.
 
 The last assertion follows from the corresponding statement on finite-dimensional $\mathfrak{sl}_2$-modules, see \ref{liestructure-subsection-representations-sl2}.
\end{proof}

Because of the shift by $\rho$ in the previous proposition, it is convenient to define a modified action of the Weyl group on $\mathfrak h^*$.

\begin{definition}
\label{definition-dot-action}
 The {\it dot action} of $W$ on $\mathfrak h^*$ is defined by 
 $$ w\bullet \lambda = w(\lambda+\rho) - \rho.$$
\end{definition}

Hence, replacing $\lambda-\rho$ by $\lambda$, the embedding of Proposition \ref{proposition-sa-quotient} reads: $M_{w_\alpha \bullet \lambda} \hookrightarrow M_\lambda$, when $\left< \lambda, \check\alpha\right> \in \mathbb Z_{\ge 0}$. 

\begin{proposition}
\label{proposition-finite-dimensional}
If $V$ is a finite-dimensional $\mathfrak g$-module, its character is $W$-invariant. Moreover, if $V$ is irreducible, it is equal to the irreducible quotient $L_\lambda$, for some weight $\lambda$ that is \emph{integral} (i.e.\ $\left< \check\alpha,\lambda\right>\in \mathbb Z$ for all roots $\alpha$) and \emph{dominant} (i.e.\ $\left< \check\alpha,\lambda\right> \ge 0$ for all positive roots $\alpha$).

Vice versa, assume that $\lambda$ is integral and dominant. Then, the representation:
$$M_{\lambda}/\left(\sum M_{w_\alpha\bullet \lambda}\right)$$
(sum over simple positive roots) is finite dimensional, and equal to the unique irreducible quotient $L_{\lambda}$ of $M_{\lambda}$. \end{proposition}

\begin{proof}
If $V$ is finite-dimensional, by Weyl's theorem \ref{liestructure-theorem-complete-reducibility} it is a direct sum of irreducibles, hence a direct sum of $L_\lambda$'s, for various weights $\lambda$. Restricting to $\mathfrak{sl}_{2,\alpha}$, the copy of $\mathfrak{sl}_2$ corresponding to a simple root $\alpha$, we see that, in order for $L_\lambda$ to be finite-dimensional, the highest weight $\lambda$ must satisfy $\left< \lambda, \check\alpha \right> \in \mathbb Z_{\ge 0}$. This holds for every $\alpha$, therefore $\lambda$ must be integral and dominant. 

In this case, by Proposition \ref{proposition-sa-quotient}, the irreducible quotient of $M_\lambda$ will have a $w_\alpha$-stable set of weights, for every simple root $\alpha$, therefore a $W$-stable set of weights. 

On the other hand, all weights are $\le \lambda$ and differ from $\lambda$ by an element of the root lattice, so there is a finite set of weights only. Finally, the weight spaces are finite dimensional, so the quotient is finite-dimensional. If the quotient were not irreducible, by complete reducibility it would be a direct sum of irreducibles, contradicting the fact that $L_\lambda$ is the unique irreducible quotient of $M_\lambda$. 
\end{proof}


\section{The Chevalley isomorphism}
\label{section-Chevalley-isomorphism}

Let $\mathfrak g$ be a reductive Lie algebra, $\mathfrak h$ a Cartan subalgebra. The goal of this section is to study the ring $\mathbb C[\mathfrak g]^{\mathfrak g}$ of invariant polynomials on $\mathfrak g$ under the adjoint representation. If $\mathfrak g$ is the Lie algebra of a connected complex group $G$, this is equal to $\mathbb C[\mathfrak g]^G$.



\begin{theorem}[Chevalley isomorphism]
 \label{theorem-Chevalley-isomorphism}
The restriction map under $\mathfrak h \hookrightarrow \mathfrak g$ gives rise to an isomorphism 
 \begin{equation}
  \label{equation-Chevalley-map}
 \text{res}:\mathbb C[\mathfrak g]^{\mathfrak g} \to \mathbb C[\mathfrak h]^W.
 \end{equation}
\end{theorem}

\begin{proof}
For injectivity, we need to use conjugacy of Cartan subalgebras, Theorem \ref{liestructure-theorem-conjugacy-Borel-Cartan}: If $G$ is the group $\mathcal E(\mathfrak g)$ of inner automorphisms of Definition \ref{liestructure-definition-Eg}, we have $\mathbb C[\mathfrak g]^{\mathfrak g}  = \mathbb C[\mathfrak g]^G$ and $\text{Ad}(G)(\mathfrak h)$ is dense in $\mathfrak g$, so the restriction map of $G$-invariant functions to $\mathfrak h$ is injective.

For $W$-invariance, it is enough to show that the image is invariant under the reflection $w_\alpha$ corresponding to any simple root $\alpha$. This simple root defines an embedding $\mathfrak m:=\mathfrak{sl}_2 \oplus \mathfrak \alpha^{\perp}\hookrightarrow \mathfrak g$, where $\alpha^\perp \subset \mathfrak h$ is the orthogonal complement of $\alpha$ in $\mathfrak h$. The Lie algebra $\mathfrak m$ contains $\mathfrak h$, and we have restriction maps 
$$\mathbb C[\mathfrak g]^{\mathfrak g} \to \mathbb C[\mathfrak m]^{\mathfrak m} \to \mathbb C[\mathfrak h].$$

This reduces us to the case of $\mathfrak{sl}_2 = \left< h , e, f\right>$.

Now, for a nilpotent element $X$ of a Lie algebra $\mathfrak g$, the automorphism $\exp(\text{ad}(X)) = \sum_{n\ge 0} \frac{1}{n!} \text{ad}(X)^n$ of $\mathfrak g$ makes sense, since $\text{ad}(X)$ is nilpotent. In the case of $\mathfrak{sl}_2$, with $w$ the nontrivial element of the Weyl group, we notice that the automorphism $w$ of $\mathfrak h$ is induced by the automorphism
$$ \tilde w = \exp(\text{ad}(e)) \exp(\text{ad}(-f)) \exp(\text{ad}(e))$$
of $\mathfrak g$. 
(This is simply conjugation by the element $\begin{pmatrix} & 1 \\ -1 & \end{pmatrix}$ of $\text{SL}_2$.) This proves that the restriction of a $\mathfrak g$-invariant polynomial function on $\mathfrak g$ to $\mathfrak h$ is $W$-invariant.


We now pass to surjectivity, which is the deepest part of the theorem. Both $\mathbb C[\mathfrak g]^{\mathfrak g}$ and $\mathbb C[\mathfrak h]^W$ are graded by the degree of a polynomial, which we will denote by an index $\mathbb C[\,\,]_d$, and the map between them preserves the grading. (We will introduce a filtration on these modules below.) Since $W$ is a finite group, the symmetrization map 
 $$ \mathbb C[\mathfrak h]_d \ni f \mapsto f_W:=\frac{1}{|W|} \sum_{w\in W} w\cdot f \in \mathbb C[\mathfrak h]_d^W$$
 is a $W$-equivariant projection, and certainly the elements of the form $\lambda^d$, $\lambda\in \mathfrak h^*$, span $\mathbb C[\mathfrak h]_d$. We can even restrict to $\lambda$ integral and dominant. 

If $\lambda$ is a dominant, integral weight, and $(\rho_\lambda, V_\lambda)$ is the irreducible finite-dimensional representation with heighest weight $\lambda$, the left hand side contains the trace function 
 $$ f_{\lambda,d}(X) = \text{tr} \rho_\lambda(X)^d.$$


We define a filtration of $\mathbb C[\mathfrak g]^{\mathfrak g}$ by dominant integral weights, where the filtered piece $F^\lambda \mathbb C[\mathfrak g]^{\mathfrak g}$ consists of the span of all $f_{\mu,d}(X)$ with $\mu\le \lambda$, where, by definition, $\mu\le \lambda \iff \lambda-\mu \in R^+$, where $R^+$ is the monoid spanned by positive roots. (This is a standard partial ordering on the weight lattice; weights which do not differ by an element in the root lattice do not have a common upper bound, but we won't worry about this since for this argument we can restrict $\lambda$ further to be in any lattice of finite index, such as the root lattice.) 

Similarly, we define a filtration of $\mathbb C[\mathfrak h]_d^W$, with $F^\lambda\mathbb C[\mathfrak h]_d^W$ spanned by the elements $(\mu^d)_W$ with $\mu\le \lambda$. Since $V_\lambda$ has a $W$-invariant set of weights, all $\le \lambda$, and with $\dim V_\lambda^\lambda=1$, we get that $\text{res}$ maps $F^\lambda \mathbb C[\mathfrak g]^{\mathfrak g} \to F^\lambda\mathbb C[\mathfrak h]_d^W$, with $\text{res}(f_{\lambda,d}) \equiv |W| (\lambda^d)_W$ in $\text{gr}^\lambda \mathbb C[\mathfrak h]_d^W$. The element $(\lambda^d)_W$ spans $\text{gr}^\lambda \mathbb C[\mathfrak h]_d^W$, thus we get by induction (the base case $\lambda = 0$ being trivial) that the map \eqref{equation-Chevalley-map} is surjective. 
\end{proof}


There is a second part to Chevalley's theorem, which asserts that the algebra of invariant functions is a polynomial algebra.

\begin{theorem}
\label{theorem-invariants-polynomial}
 Let $E$ be a complex vector space, and $W\subset \text{GL}(E)$ a finite subgroup of automorphisms, which is generated by pseudoreflections (i.e., elements that fix a hyperplane). Then, the algebra $\mathbb C[E]^W$ is a polynomial algebra in $d=\dim(E)$ generators. In particular, for a semisimple Lie algebra $\mathfrak g$ over $\mathbb C$, the algebra of invariants $\mathbb C[\mathfrak g]^{\mathfrak g}$ is a polynomial algebra over $\mathbb C$ in $\text{rank}(\mathfrak g)$ generators.
\end{theorem}


\begin{proof}
 See \cite{Sternberg}, for now. 
\end{proof}

\begin{example}
\label{example-invariants-sl2}
 For $\mathfrak g= \mathfrak{sl}_n$, the coefficients of the characteristic polynomial of an element $X$:
 $$ \chi_X(t) = t^n + \text{tr}(-X; \wedge^2 \mathbb C^n) t^{n-2} + \dots + \text{tr}(-X; \wedge^{n-1} \mathbb C^n) t + \det(-X)$$
 generate the ring $\mathbb C[\mathfrak g]^{\mathfrak g}$ freely.
\end{example}


\begin{definition}
 \label{definition-fundamental-invariants}
 The {\it fundamental degrees} of a finite reflection group $W$ acting on a Euclidean space $E$ are the degrees of a set of homogeneous free generators of the polynomial ring $\mathbb C[E]^W$. The fundamental degrees of a semisimple Lie algebra are the fundamental degrees of its root system.
\end{definition}

For example, for $\mathfrak g=\mathfrak{sl}_n$, the fundamental degrees are $2, \dots, n$.
These degrees ($d_i$, $i=1, \dots, \dim E$) are uniquely defined, and have some interesting properties, for example: 
\begin{equation}
 \label{equation-degrees-product}
 \prod_i d_i = |W|,
\end{equation}
\begin{equation}
 \label{equation-degrees-sum}
 \sum_i d_i = \frac{|\Phi|}{2} + \dim E.
\end{equation}
See \cite[\S 3]{Humphreys-reflection}.


\section{The Harish-Chandra homomorphism}
\label{section-HC-homomorphism}

\begin{definition}
 \label{definition-HC-center}
The {\it Harish-Chandra center} of a Lie algebra $\mathfrak g$ is the center $Z(\mathfrak g)$ of the universal enveloping algebra $U(\mathfrak g)$.
\end{definition}

Notice that, although we write $Z(\mathfrak g)$, this is not the center of $\mathfrak g$ itself (which is trivial for semisimple algebras), but of its universal enveloping algebra. 

What goes by the name of ``Harish-Chandra homomorphism'' is actually an isomorphism, between $Z(\mathfrak g)$ and the polynomial ring $\mathbb C[\mathfrak h^*]^{W,\bullet} = U(\mathfrak h)^{W,\bullet}$, where $\bullet$ denotes the dot action of $W$, see \ref{definition-dot-action}. To construct it, we consider the action of $Z(\mathfrak g)$ on a specific $\mathfrak g$-module with a commuting $\mathfrak h$-action. 


\begin{definition}
 \label{definition-universal-Verma-module}
The {\it universal Verma module} is the $\mathfrak g$-module $M$ with the property that
$$\Hom_{\mathfrak g}(M, V) = \Hom_{\mathfrak n} (\mathbb C,V).$$
\end{definition}

As with Verma modules, the universal Verma module $M$ exists, and can be identified with
$$ M= U(\mathfrak g)\otimes_{U(\mathfrak n)} \mathbb C = U(\mathfrak g)/U(\mathfrak g)\mathfrak n.$$

Writing 
$$M=U(\mathfrak g)\otimes_{U(\mathfrak n)} \mathbb C = U(\mathfrak g)\otimes_{U(\mathfrak b)}( U(\mathfrak b)\otimes_{U(\mathfrak n)} \mathbb C ) = U(\mathfrak g)\otimes_{U(\mathfrak b)} U(\mathfrak h),$$ 
we see that $M$ is a $\mathfrak g\times\mathfrak h$-module. The following lemma will give rise to the Harish-Chandra homomorphism:

\begin{proposition}
\label{proposition-Zg-Zh}
 For every $X\in Z(\mathfrak g)$, there is a unique element $\phi(X)\in U(\mathfrak h) $ such that the action of $X$ on the universal Verma module $M$ coincides with the action of $\phi(X)$. The resulting map 
 $$ \phi: Z(\mathfrak g)\to U(\mathfrak h)$$
 is a ring homomorphism.
\end{proposition}

\begin{remark}
 \label{remark-Zg-Zh-explicit}
Explicitly, Proposition \ref{proposition-Zg-Zh} says that, under the identification $M= U(\mathfrak g)/U(\mathfrak g)\mathfrak n$, the image of $Z(\mathfrak g)$ in the quotient lies in the image of $U(\mathfrak h)$; in other words, $Z(\mathfrak g)\subset U(\mathfrak h) + U(\mathfrak g)\mathfrak n$. 
\end{remark}


Notice that $U(\mathfrak h) = S(\mathfrak h) = \mathbb C[\mathfrak h^*]$. 

\begin{proof}
 The action of $Z(\mathfrak g)$ commutes with that of $U(\mathfrak g)$, so it suffices to show that the action of $X$ on a generator of the module $M$ coincides with the action of some $\phi(X)\in U(\mathfrak h)$. Take this generator to be the element $1:=1\otimes 1 \in  U(\mathfrak g)\otimes_{U(\mathfrak b)} U(\mathfrak h)$. This element is annihilated by the adjoint action of $\mathfrak g$, in particular, under the action of $\mathfrak h$ considered as a subalgebra of $\mathfrak g$. Using the Poincar\'e--Birkhoff--Witt theorem, we compute that $M$, restricted to $\mathfrak h\subset \mathfrak g$, is isomorphic to 
 $$U(\mathfrak n^-) \otimes  U(\mathfrak h).$$
 Since $U(\mathfrak n^-)^{\mathfrak h} = \mathbb C$, we get that $M^{\mathfrak h} =U(\mathfrak h)$, so $X\cdot 1 = \phi(X)\in U(\mathfrak h)$. 
 
 But $\phi(X)$ is also the image of the element $\phi(X)\in U(\mathfrak h)$ acting on $1$ via the action of $\mathfrak h$ that commutes with the action of $\mathfrak g$, which we will denote as a right action:
 $$ \phi(X) = 1\cdot \phi(X).$$
 Since the action of $U(\mathfrak g)$ commutes with the action of $Z(\mathfrak g)$, the same holds when we replace $1$ by any element $Z\in M$:
 $$ X\cdot M = M\cdot \phi(X).$$
 Therefore $\phi(XY) = 1\cdot \phi(XY) = XY \cdot 1 = X(1\cdot \phi(Y)) = (1\cdot \phi(Y))\cdot \phi(X) = 1\cdot \phi(X)\phi(Y)$, and the map $\phi$ is a homomorphism.
\end{proof}

\begin{definition}
 \label{definition-HC-homomorphism}
The homomorphism $\phi: Z(\mathfrak g)\to U(\mathfrak h)$ of Proposition \ref{proposition-Zg-Zh} is the {\it Harish-Chandra homomorphism}.
\end{definition}

\begin{proposition}
\label{proposition-HCimage-Winvariant}
 The center $Z(\mathfrak g)$ of $U(\mathfrak g)$ acts on each Verma module $M_\lambda$ by a character $\chi_\lambda: Z(\mathfrak g)\to \mathbb C$. If $\lambda, \mu$ are integral and conjugate by the dot action of the Weyl group (Definition \ref{definition-dot-action}), then $\chi_\lambda = \chi_\mu$. 
\end{proposition}

\begin{proof}
 The center preserves the eigenspaces for the $\mathfrak h$-action, and since $M_\lambda^\lambda$ is one-dimensional, it acts on it by a scalar. This generates $M_\lambda$ under the $U(\mathfrak g)$-action, so the center acts by the same scalar on all of $M_\lambda$.
 
 If $\lambda, \mu$ are integral, $\lambda$ is dominant, and $w\bullet \mu = \lambda$  for some $w\in W$, we saw in Proposition \ref{proposition-sa-quotient} that there is an embedding $M_\mu \hookrightarrow M_\lambda$, therefore $\chi_\mu=\chi_\lambda$. Since $w$ is arbitrary, the same holds without the assumption that $\lambda$ be dominant. 
\end{proof}



\begin{theorem}
\label{theorem-HC-isomorphism}
 The Harish-Chandra homomorphism is injective, and gives rise to an isomorphism
 $$ Z(\mathfrak g) \simeq \mathbb C[\mathfrak h^*]^{W, \bullet},$$
 where the exponent on the right means invariants with respect to the dot action \ref{definition-dot-action}.
\end{theorem}

 Of course, the map $\lambda\mapsto \rho+\lambda$ induces, by pullback, an isomorphism $\mathbb C[\mathfrak h^*]^W \simeq \mathbb C[\mathfrak h^*]^{W,\bullet}$, but it is good to keep in mind that the most natural map gives rise to invariants with respect to the dot action.

\begin{proof}
Any $\lambda\in \mathfrak h^*$ defines a morphism of $\mathfrak b$-modules
$U(\mathfrak h)\to \mathbb C_\lambda$, which by induction gives rise to a morphism $M\to M_\lambda$. Therefore, the character $\chi_\lambda$ by which $Z(\mathfrak g)$ acts on $M_\lambda$ is equal to $\lambda \circ \phi$, where $\phi$ is the Harish-Chandra homomorphism. For every integral $\lambda$, and every $w\in W$, we have, by Proposition \ref{proposition-HCimage-Winvariant}, $\lambda \circ \phi = (w\bullet \lambda)\circ\phi$, and since those $\lambda$'s are Zariski dense in $\mathfrak h^*$, the image of the Harish-Chandra homomorphism lies in the invariants for the dot action. 

Having constructed the homomorphism $Z(\mathfrak g) \to \mathbb C[\mathfrak h^*]^{W,\bullet}$, there remains to show that it is a bijection. The argument to be used is quite general: once a homomorphism between filtered rings is constructed, to show that it is an isomorphism, it is enough to show this for their associated graded rings. 

Notice that the natural filtration of $U(\mathfrak g)$ is $\mathfrak g$-stable, and therefore 
$$ Z(\mathfrak g) = \underset{\to}{\lim} (F^d U(\mathfrak g))^{\mathfrak g}.$$
Thus, $Z(\mathfrak g)$ is filtered by $F^d Z(\mathfrak g)= (F^d U(\mathfrak g))^{\mathfrak g}$. The Harish-Chandra homomorphism respects this filtration, and induces a homomorphism of the associated graded:
$$ \text{gr} \phi : \text{gr} Z(\mathfrak g) \to \text{gr} U(\mathfrak h)^{W,\bullet}.$$
Also, notice that the shift by $\rho$ in the definition of the dot action of $W$ is not seen at the graded level, so $\text{gr} U(\mathfrak h)^{W,\bullet}$ is canonically equal to $S(\mathfrak h)^W$ (invariants for the standard action of $W$). 

In what follows, we will use an invariant bilinear form to identify $\mathfrak g \simeq \mathfrak g^*$, $\mathfrak h = \mathfrak h^*$, and apply the Chevalley isomorphism of Theorem \ref{theorem-Chevalley-isomorphism}. In doing so, we keep in mind that the invariant bilinear form identifies $\mathfrak n$ as the orthogonal complement of $\mathfrak b$. Therefore, the restriction map $S(\mathfrak g)=\mathbb C[\mathfrak g^*] \to \mathbb C[\mathfrak h^*] = S(\mathfrak h)$ takes the ideal $S(\mathfrak g)\mathfrak n$ to zero.

By Remark \ref{remark-Zg-Zh-explicit}, any element in $Z(\mathfrak g)$ belongs to the subspace $U(\mathfrak h) \oplus U(\mathfrak g)\mathfrak n \subset U(\mathfrak g)$. Restricting to the $d$-th piece of the filtration, we obtain a commutative diagram

[Look at pdf file if diagram does not appear.]

\begin{xy}
\xymatrix{
F^d Z(\mathfrak g) \ar[d]^{HC} \ar@{^{(}->}[r] &  \ar[dl]_{\text{proj}} F^d\left( U(\mathfrak h) \oplus U(\mathfrak g)\mathfrak n\right)  \ar[d]^{\text{gr}} \ar@{^{(}->}[r] &  F^d U(\mathfrak g) \ar[d]^{\text{gr}} \\
F^d U(\mathfrak h)\ar[dr]^{\text{gr}} & S^d(\mathfrak h) \oplus S^{d-1}(\mathfrak g)\mathfrak n \ar[d]^{\text{proj}} \ar@{^{(}->}[r] & S^d(\mathfrak g) \ar[dl]^{\text{res}} \\
& S^d(\mathfrak h).
}
\end{xy}

The composition $F^d Z(\mathfrak g) \to S^d(\mathfrak h)$ is precisely the grading of the Harish-Chandra homomorphism $\text{gr}^d \phi: \text{gr}^d Z(\mathfrak g)\to \text{gr}^d U(\mathfrak h)^{W,\bullet}$, as can be seen by following the arrows on the left. 

On the other hand, because of complete reducibility (Theorem \ref{liestructure-theorem-complete-reducibility}), the associated graded of $Z(\mathfrak g)$ is 
$$\text{gr}^d Z(\mathfrak g) = \left(\text{gr}^d U(\mathfrak g)\right)^{\mathfrak g}  = S^d(\mathfrak g)^{\mathfrak g},$$
where we have used complete reducibility to say that the functor of $\mathfrak g$-invariants, applied to the short exact sequence
$$ 0 \to F^{d-1} U(\mathfrak g) \to F^d U(\mathfrak g) \to S^d(\mathfrak g) \to 0,$$
preserves exactness.

Thus, applying the functor of $\mathfrak g$-invariants on the right-most arrows in the diagram, and the Chevalley isomorphism \ref{theorem-Chevalley-isomorphism}, we obtain that $\text{gr}^d \phi$ is an isomorphism: $\text{gr}^d Z(\mathfrak g)\to S(\mathfrak h)^W$. Thus, the Harish-Chandra homomorphism $\phi$ is an isomorphism onto $U(\mathfrak h)^{W,\bullet} = \mathbb C[\mathfrak h^*]^{W,\bullet}$.

\end{proof}







\section{Localization with respect to $Z(\mathfrak g)$}
\label{section-localization}

We return to the study of the category $\mathcal O$. 

\begin{lemma}
\label{lemma-locally-finite}
For every object $V$ in $\mathcal O$, the action of $Z(\mathfrak g)$ on $V$ is locally finite.
\end{lemma}

\begin{proof}
We have seen in Lemma \ref{lemma-filtration-Verma} that every object can be filtered by surjective images of Verma modules. By Proposition \ref{proposition-HCimage-Winvariant}, $Z(\mathfrak g)$ acts by a scalar on Verma modules, hence also on their quotients. Therefore, it acts locally finitely on finite extensions of such objects.
\end{proof}



Set $\mathfrak h^*//_\bullet W = \text{Spec} \mathbb C[\mathfrak h^*]^{W,\bullet}$, so that the maximal ideals of $Z(\mathfrak g)$ are the complex points of the quotient $\mathfrak h^*//_\bullet W$, which are the points of the set-theoretic quotient of $\mathfrak h^*$ by the dot action of $W$.  

For $\chi\in \mathfrak h^*//_\bullet W$, we let $\mathcal O_\chi$ denote the full subcategory consisting of objects of $\mathcal O$ which are generalized eigenspaces for $Z(\mathfrak g)$ with generalized eigencharacter $\chi$. 

\begin{theorem}
\label{theorem-decomposition-O}
 \begin{enumerate}
  \item The category $\mathcal O$ is a direct sum of categories $\mathcal O_\chi$, with $\chi$ varying over the complex points of $\mathfrak h^*//_\bullet W$.
  \item If $\lambda$ is such that $\lambda-w\bullet \lambda$ is not, for any element $w\in W$, in the positive root monoid $R^+=\{\sum_{\alpha\in\Phi^+} n_\alpha \alpha | n_\alpha\in\mathbb N\}$ and nonzero, then $M_\lambda$ is irreducible.
  \item Every object in $\mathcal O$ is of finite length. 
  \item The classes of the Verma modules $M_\lambda$ (or, equivalently, their irreducible quotients $L_\lambda$) are a basis for the Grothendieck group $\mathbb Z[\mathcal{O}]$.
 \end{enumerate}
\end{theorem}


\begin{proof}
 \begin{enumerate}
  \item Since the action of the center is locally finite, we can decompose every object into a direct sum of generalized $Z(\mathfrak g)$-eigenspaces. Any $\mathfrak g$-morphism commutes with the action of $Z(\mathfrak g)$, so there are no $\mathfrak g$-morphisms between them.
  \item If $M_\lambda$ is not irreducible, it contains a highest weight vector of weight $\mu<\lambda$ (in the same partial ordering as previously, i.e., $\mu\le \lambda$ means that $\lambda -\mu \in R^+$, the positive root monoid), hence there is a nontrivial map from $M_\mu$ to $M_\lambda$. By the decomposition of the category, on the other hand, such a nontrivial map can exist only if $\mu=w\bullet \lambda$ for some $w\in W$.
  \item By Lemma \ref{lemma-filtration-Verma}, it suffices to show that Verma modules are of finite length. Let $K_\lambda$ be the kernel of the map $M_\lambda\to L_\lambda$.
If nonzero, then $K_\lambda$ admits a filtration as in Lemma \ref{lemma-filtration-Verma}, whose factors are surjective images of modules $M_\mu$ with $\mu<\lambda$. But by the decomposition of categories, $\mu$ has to be a $W$-conjugate of $\lambda$ (for the dot action), hence after a finite number of steps the module $M_\mu$ will be irreducible, by the previous statement. 
  \item By the same argument, every object can be filtered by successive quotients of Verma modules, so they generate the Grothendieck group of the category. There cannot be a nontrivial relation between them,
  $$ \sum n_i [M_{\lambda_i}]=0,$$
because for any $\lambda=\lambda_i$ which is maximal among the $\lambda_i$'s in the partial ordering of weights, the $\lambda$-weight space of $M_\lambda$ cannot be cancelled by another term. The fact that the simple modules $L_\lambda$ also form a basis follows from the fact that the category is Artinian (every object is of finite length), and they are the only irreducible objects (non-isomorphic to each other). 
\end{enumerate}
\end{proof}


As before, let $C$ be the ring of formal sums $\sum_{\lambda\in \mathfrak h^*} c(\lambda) e^\lambda$, where $c_\bullet: \mathfrak h^*\to \mathbb Z$ is supported in a finite number of translates of the negative root monoid, and multiplication defined by $e^\lambda \cdot e^\mu = e^{\lambda+\mu}$. 

\begin{definition}
 \label{definition-Weyl-denominator}
The {\it Weyl denominator} is the following element of $C$: 
$$\Delta = \prod_{\alpha>0} \left( e^\frac{\alpha}{2} - e^{-\frac{\alpha}{2}}\right) =  e^\rho \prod_{\alpha>0} \left( 1 - e^{-{\alpha}}\right)$$
(the product over all positive roots), thought of as a power series in elements of $\rho-R^+$ (where $R^+$ is the positive root monoid).
\end{definition}

Notice that the weight $\rho$ is integral because $\alpha = \rho- w_\alpha\rho = \left< \rho,\check\alpha\right> \alpha$ for every simple root $\alpha$. The name ``Weyl denominator'' is due to its appearance in the Weyl character formula, Theorem \ref{theorem-Weyl-character}.


\begin{proposition}
\label{proposition-character-Verma}
 The character of the Verma module $M_\lambda$ satisfies:
$$ \Delta \cdot \text{ch}_{M_\lambda} = e^{\lambda+\rho}.$$
\end{proposition}

\begin{proof}
 As an $\mathfrak h$-module, $M_\lambda = U(\mathfrak n_-) \otimes \mathbb C_\lambda$, so $\text{ch}(V) = \text{ch}(U(\mathfrak n_-))\cdot \text{ch}(\mathbb C_{\lambda}) = \chi(U(\mathfrak n^-))\cdot e^{\lambda}$. Therefore, it suffices to prove that the character of $U(\mathfrak n_-)$ is $e^\rho/L$, understood as a power series in the \emph{negative} weight monoid.

 By the Poincar\'e--Birkhoff--Witt theorem \ref{liegroups-general-theorem-PBW}, as $\mathfrak h$-modules we have: $U(\mathfrak n_-) = \otimes_{\alpha>0} S(\mathfrak g_{-\alpha})$. The character of $S(\mathfrak g_{-\alpha})$ is $1+e^{\alpha}+e^{2\alpha}+\dots = \frac{1}{1-e^{-\alpha}}$, and this proves the proposition.
\end{proof}

Finally, we are ready to prove the \emph{Weyl character formula}:

\begin{theorem}
\label{theorem-Weyl-character}
The character of the irreducible representation with heighest weight $\lambda$ is given by the \emph{Schur polynomial}:
$$ \text{ch}_{V_\lambda} = \frac{\sum_{w\in W} \text{sgn}(w) e^{w(\lambda+\rho)}}{\Delta}$$
\end{theorem}

\begin{proof}
By Proposition \ref{proposition-finite-dimensional} and Theorem \ref{theorem-decomposition-O}, we have $V_\lambda=L_\lambda$, and an equality of the form  
$$[V_\lambda]= [M_{\lambda}] + \sum_{w\in W, w\ne 1} c_w [M_{w\bullet \lambda}]$$
in the Grothendieck group, for some integers $c_w$. Indeed, Proposition \ref{proposition-finite-dimensional} represents $L_\lambda$ as a quotient of $M_\lambda$ by the image of a morphism 
$$\bigoplus_{w\ne 1} M_{w\bullet \lambda} \to M_\lambda,$$
and by Theorem \ref{theorem-decomposition-O} the kernel of this morphism will have a finite composition series in terms of the $M_{w\bullet \lambda}$'s, necessarily with $w\ne 1$ as the weight $\lambda$ does not appear in the kernel.

Hence,
$$ \Delta \cdot \text{ch}(V_\lambda) = e^{\lambda+\rho} + \sum_{w\in W,w\ne 1} c_w e^{w(\lambda+\rho)}.$$

On the other hand, again by Proposition \ref{proposition-finite-dimensional}, the character is $W$-invariant. Therefore the expression on the right should be $(W,\text{sgn})$-equivariant. Therefore, $c_w = \text{sgn}(w)$.
\end{proof}



\section{Example: Irreducible representations of $\mathfrak{sl}_n$. Schur--Weyl duality.}
\label{section-Schur-Weyl}

Schur--Weyl duality refers to a correspondence between representations of symmetric groups and general linear groups (or their Lie algebras), which is realized inside the tensor powers $V^{\otimes^d}$ of a vector space. It is based on the following theorem from linear algebra:

\begin{theorem}
\label{theorem-double-centralizer}
If $V$ is a finite-dimensional complex vector space, $A\subset \text{End}_B(V)$ is a semisimple subalgebra of operators, and $B = \text{End}_A(V)$ is its commutant, then 
\begin{enumerate}
 \item $B$ is semisimple.
 \item $A=\text{End}_B(V)$.
 \item There is a bijection $M_i\leftrightarrow N_i$ between isomorphism classes of simple $A$-modules and isomorphism classes of simple $B$-modules, and an isomorphism of $A\otimes B$-modules 
 $$V = \bigoplus_i M_i\otimes N_i.$$
\end{enumerate}
\end{theorem}

\begin{proof}
The proof uses the Artin--Wedderburn theorem which, in the case of the complex numbers, says that a finite-dimensional complex semisimple algebra is the direct sum of the endomorphism algebras of its simple modules. 

Let $M_i$ range over all isomorphism classes of simple $A$-modules. Since $V$ is $A$-semisimple, 
 \begin{equation}
  \label{equation-double-centralizer}
  V = \bigoplus_i M_i\otimes \text{Hom}_A(M_i, V).
 \end{equation}
 Set $N_i=\text{Hom}_A(M_i, V)$. Since the $M_i$'s are non-isomorphic, we have, by Schur's lemma, 
 $$ B=\text{End}_A(V) = \bigoplus_i \text{End}(N_i).$$
 The unique isomorphism class of simple $\text{End}(N_i)$-modules is $N_i$, hence $B$ is semisimple, with its simple modules being precisely the $N_i$'s, which are non-isomorphic. Applying now the same reasoning to \eqref{equation-double-centralizer}, we see that 
 $$ \text{End}_B(V) = \bigoplus_i \text{End}(M_i).$$
 But $A$ being semisimple means that it is isomorphic to $\bigoplus_i \text{End}(M_i)$, thus $\text{End}_B(V)=A$ and the map $M_i\mapsto N_i$ is a bijection of isomorphism classes of simple modules.
\end{proof}


Now we apply this to $S_d$ and $\mathfrak{gl}(V)$:


\begin{theorem}[Schur--Weyl duality]
\label{theorem-Schur-Weyl-centralizer}
 Consider the space $V^{\otimes^d}$ under the commuting actions of $S_d$ and $\mathfrak{gl}(V)$, i.e., as a representation of the algebra $A\otimes B$, where $A=\mathbb C[S_d]$ and $B=U(\mathfrak{gl}(V))$. Then, the images $\bar A$, $\bar B$ of $A$ and $B$ in $\text{End}(V^{\otimes^d})$ are each others' commutants, that is, 
 $$\bar A=\text{End}_B(V^{\otimes^d}),\mbox{ and }$$
 $$\bar B=\text{End}_A(V^{\otimes^d}).$$
 
 We have a decomposition 
\begin{equation}
 \label{equation-Schur-Weyl}
 V^{\otimes^d} = \bigoplus_{\tau} \tau \otimes \theta(\tau),
\end{equation}
 where $\tau$ ranges all isomorphism classes of irreducible representations of $S_d$, and the $\theta(\tau):= \text{Hom}_{S_d} (\tau, V^{\otimes^d})$ are either zero, or distinct irreducible representations of $\mathfrak{gl}(V)$.
\end{theorem}

Notice that the action of $\mathfrak{gl}(V)$ on $V^{\otimes^d}$ is defined as we define tensor products of representations of Lie algebras, i.e., the image of $e\in \mathfrak{gl}(V)$ in $\text{End}(V^{\otimes^d})$ is the element $S_d e:=\sum_{i=1}^d 1\otimes\cdots\otimes e \mbox{ ($i$-th factor)} \otimes \cdots\otimes 1$.

\begin{proof}
 Since both subalgebras are semisimple (complete reducibility), by Theorem \ref{theorem-double-centralizer} it is enough to prove the second claim.

 We have $\text{End}_A(V^{\otimes^d}) = \text{End}(V^{\otimes^d})^{S_d} = (\text{End}(V)^{\otimes^d})^{S_d} = S^d \text{End}(V)$, and the $d$-th symmetric power of any vector space $E$ is spanned by the symmetric tensors $e\otimes \dots \otimes e$, for $e\in E$. In this case, $E=\text{End}(V)=B$. By the theory of symmetric polynomials, $e\otimes \dots \otimes e$ is a polynomial in the elements $S_d(e^i)$, $i=1,\dots, d$, which are in the image of $\mathfrak{gl}(V)$.
\end{proof}


Finally, we notice

\begin{lemma}
\label{lemma-restriction-sl}
Every irreducible representation of $\mathfrak{gl}_n$ restricts irreducibly to $\mathfrak{sl}_n$.
\end{lemma}

\begin{proof}
 We have $\mathfrak{gl}_n = \mathfrak z \oplus \mathfrak{sl}_n$, where $\mathfrak z$ is the center, but the center acts by a scalar, by Schur's lemma, so any $\mathfrak{sl}_n$-invariant subspace is also $\mathfrak{gl}_n$-invariant.
\end{proof}

Therefore, the irreducible representations of $\mathfrak{gl}_n$ constructed in Theorem \ref{theorem-Schur-Weyl} are also irreducible for $\mathfrak{sl}_n$. We will now classify irreducible representations of the symmetric group, and make the correspondence explicit, observing, in particular, that when $d$ is large enough, the decomposition \ref{equation-Schur-Weyl} contains all irreducible representations of $\mathfrak{sl}_n$.


We follow, and reformulate, \cite{Fulton-Harris}, where we point the reader for more fun, to describe irreducible representations of the symmetric group $S_d$. 

For the Lie algebra $\mathfrak{gl}_n$ with the standard Cartan of diagonal elements and the standard Borel of upper triangular elements, the dominant, integral weights are of the form
$$ \text{diag}(z_1, z_2, \dots, z_n)\mapsto \lambda_1 z_1 + \dots +\lambda_n z_n,$$
with $\lambda_1\ge \lambda_2\ge \dots \ge \lambda_n$ some integers. 

For the Lie algebra $\mathfrak{sl}_n$, the positive, integral weights are described similarly, except that the $\lambda_i$'s are determined modulo the operation of adding the same constant to all of them. To reduce ambiguity, we can always take $\lambda_n=0$ (but won't be doing that yet).

On the other hand, if $\lambda_1\ge \lambda_2\ge \dots \ge \lambda_n \ge 0$ are integers, and $d=\sum_i \lambda_i$, the $\lambda_i$'s describe a conjugacy class in the symmetric group $S_d$, namely, the conjugacy class of elements which can are products of disjoint cycles of lengths $\lambda_1$, $\lambda_2, \dots, \lambda_n$. Notice that $\lambda_n$ has been taken to be $\ge 0$ here, which is slightly restrictive for $\mathfrak{gl}_n$, but not for $\mathfrak{sl}_n$; in fact, for $\mathfrak{sl}_n$, any integral, dominant weight defines a conjugacy class in any $S_{d+ kn}$, for the minimal $d$ determined by $\lambda_n=0$. 

We consider $G=S_d$ as the permutation group on the set $\Sigma = \{1,\dots, d\}$, and for every $\lambda:\lambda_1\ge \lambda_2\ge \dots \ge \lambda_n \ge 0$, a $\lambda$-partition of $\Sigma$ will be a disjoint decomposition $\Sigma = \bigsqcup_i \Sigma_i$, with $|\Sigma_i| = \lambda_i$. We can think of $\lambda$ as a \emph{Young diagram}, that is, the diagram consisting of a row of $\lambda_1$ squares stacked over $\lambda_2$ squares (aligned on the left), etc,\footnote{Since we allow some $\lambda_i$'s to be zero, the Young diagram does not determine $\lambda$, unless $n$ is known, but this will not play a role in what follows.} and we can also think of \emph{Young tableaux}, which are ways to populate the squares of a given Young diagram with the elements of $\Sigma$ (without repetitions). Then, the group $G=S_d$ acts on the space of Young tableaux of a given shape $\lambda$ (we define this action as a right action), and the space $\Sigma_\lambda$ of $\lambda$-partitions of $\Sigma$ is the homogeneous space $P\backslash G$, where $P=G_\lambda$ is the stabilizer of the rows of the standard Young tableau (where the integers are placed in order). 

The \emph{dual partition} to $\lambda$ is partition $\lambda^*:\lambda^*_1\ge \lambda^*_2 \ge \dots \lambda^*_m >0$ of $d$ counting the sizes of the columns of the Young diagram of $\lambda$. The space $\Sigma_{\lambda^*}$ of $\lambda^*$-partitions of $\Sigma$ is the homogeneous space $Q\backslash G$, where $Q=G_{\lambda^*}$ is the stabilizer of the columns of the standard Young tableau.

We will construct the irreducible representations of $G$ by inducing the trivial and sign representation from the groups $P$ and $Q$. None of them is irreducible, but they share a unique irreducible component.





The double quotient space $P\backslash G/Q$ can be identified with $(\Sigma_\lambda \times \Sigma_\mu)/G^{\text{diag}}$, i.e., with the set of pairs
$(\sigma,\tau)$, where $\sigma$ is a $\lambda$-partition of $\Sigma$ and $\tau$ is a $\mu$-partition of $\Sigma$, \emph{up to relabeling the elements of $\Sigma$}.

The classification of irreducible representations of $S_d$ rests upon the following fundamental lemma:

\begin{lemma}
\label{lemma-combinatorics}
 If $\lambda, \mu$ are two partitions of $d$, and $(\sigma,\tau)$ is a pair consisting of a $\lambda$-partition $\sigma$ of $\Sigma$ and a $\mu^*$-partition $\tau$ of $\Sigma$ with no pair $(k,l)$ of elements of $\Sigma$ in the same subset of $\sigma$ and of $\tau$, then $\tau$ is the refinement of a $\lambda^*$-partition of $\Sigma$; in particular, $\mu \ge \lambda$ in the lexicographic ordering, i.e., $\mu=\lambda$ or at the first index $i$ where $\mu_i\ne \lambda_i$ we have $\mu_i > \lambda_i$.
 
 If $\mu=\lambda$ then the only pairs $(\sigma,\tau)$ without a pair $(k,l)$ in the same subset of $\sigma$ and of $\tau$ are those where $\sigma, \tau$ are the rows, resp.\ columns, of a single Young tableau.
\end{lemma}


\begin{proof}
 Exercise.
\end{proof}

Now, we let $M_\lambda$, resp.\ $A_\lambda$, be the $G$-equivariant line bundles over the space $\Sigma_\lambda$ which are induced, respectively, from the trivial, resp.\ sign character, of $P$. Explicitly, sections of $M_\lambda$ are left-$P$-invariant functions on $G$, i.e., left-$P$-invariant elements of $\mathbb C[G]$, while sections of $A_\lambda$ are functions on $G$ which vary by the sign character under left translation by $P$. For notational simplicity, we will identify the bundles with their space of sections.

\begin{proposition}
 \label{proposition-intertwiners}
We have $\dim\Hom_G(M_\lambda, A_{\lambda^*}) = 1$. 

If $\lambda>\mu$, we have $\dim\Hom_G(M_\lambda, A_{\mu^*}) = 0$.
\end{proposition}

\begin{proof}
The space of $G$-morphisms $\Hom_G(M_\lambda, A_{\mu^*})$ can naturally be identified with the space of $G^{\text{diag}}$-invariant sections of $L:=M_\lambda\otimes A_{\mu^*}$ over $\Sigma_\lambda\times\Sigma_{\mu^*}$, considered as kernel functions, i.e., the morphism $T_K$ corresponding to a section $K$ is
$$ T_K (f) (y) = \sum_{x\in \Sigma_\lambda} f(x) K(x,y).$$

By Lemma \ref{lemma-combinatorics}, if $\lambda \ge \mu$, for a pair $(\sigma,\tau) \in \Sigma_\lambda \times \Sigma_{\mu^*}$ there is a transposition $t=(k,l)$ which stabilizes both $\sigma$ and $\tau$, unless $\lambda=\mu$ and the pair $(\sigma,\tau)$ corresponds to the rows and columns of a Young tableau. But then, $t$ will act by $-1$ on the fiber of $L$ over $(\sigma,\tau)$, which means that its orbit cannot support a $G$-invariant section of $L$. 
\end{proof}


\begin{theorem}
\label{theorem-irreducibles-symmetric}
The image of a nonzero $G$-morphism $M_\lambda \to A_{\lambda^*}$ is an irreducible representation $V_\lambda$. For $\lambda, \mu$ different partitions, $V_\lambda, V_\mu$ are non-isomorphic, and these are all the irreducible representations of $G=S_d$.
\end{theorem}

\begin{proof}
The image of a nonzero $G$-morphism $M_\lambda \to A_{\lambda^*}$ has to be an irreducible representation $V_\lambda$, because otherwise, the space $\Hom(M_\lambda, A_{\lambda^*})$ would have dimension $>1$, by scaling the irreducible summands in the image by different scalars. This would contradict Proposition \ref{proposition-intertwiners}.
 
Let $\lambda\ne \mu$. Without loss of generality, assume that $\lambda >\mu$ in the lexicographic order. Then, again by Proposition \ref{proposition-intertwiners}, there are no $G$-morphisms from $M_\lambda$ to $A_{\mu^*}$. Therefore, $V_\lambda$ cannot embed in $A_{\mu^*}$, and is non-isomorphic to $V_\mu$. 

The number of partitions $\lambda$ of $d$ is equal to the number of conjugacy classes of $S_d$, so we have constructed all the isomorphism classes of irreducible $S_d$-representations.
\end{proof}





Explicitly, by the bijection between $P\backslash G/Q$ and $(\Sigma_\lambda \times \Sigma_\lambda^*)/G^{\text{diag}}$, we can think of a $G$-invariant section of $L$ as an element of $\mathbb C[G]$ which is left-$P$-invariant and varies by the sign character under right multiplication by $Q$. Then, a basis element for the space of invariant sections is the element 
$$ c_\lambda = a_\lambda b_\lambda,$$
where 
$$ a_\lambda = \sum_{g\in P} g,$$
$$ b_\lambda = \sum_{g\in Q} \text{sgn}(g)\cdot g.$$


\begin{lemma}
\label{lemma-projectors}
Let $a_\lambda, b_\lambda, c_\lambda \in \mathbb C[S_d]$ be as above. Then, the irreducible representation $V_\lambda$ of $S_d$ is isomorphic to the module $\mathbb C[S_d] c_\lambda$, or equivalently to the module $c_\lambda^* \mathbb C[S_d]$, where $c_\lambda^*= b_\lambda a_\lambda$. 

For any vector space $V$, the action of $a_\lambda$ induces a surjective map onto the tensor product of symmetric powers
$$V^{\otimes^d} \to S^{\lambda_1}V \otimes \cdots \otimes S^{\lambda_n} V,$$
while the action of $b_\lambda$ induces a surjective map onto the tensor product of exterior powers
$$V^{\otimes^d} \to \bigwedge^{\lambda_1^*}V \otimes \cdots \otimes \bigwedge^{\lambda_m^*} V.$$

Moreover:
\begin{enumerate}
 \item $a_\lambda \cdot x \cdot b_\nu = 0$ whenever $\nu: \nu_1 \ge \dots \nu_r \ge 0$ is a partition of $d$ which is \emph{smaller} than $\lambda$ in the lexicographic ordering, i.e., $\nu_i = \lambda_i$ for some $j$ and all $i<j$, while $\lambda_j>\nu_j$. 
 \item $c_\lambda$ is the only element $c\in \mathbb C[S_d]$, up to scalar, with the property that $p c q = \text{sgn}(q) c$ for all $p\in P$, $q\in Q$. 
 \item $c_\lambda x c_\lambda$ is a multiple of $c_\lambda$, for every $x\in \mathbb C[S_d]$. In particular, $c_\lambda$ is an idempotent up to a scalar, i.e., $c_\lambda^2 = n_\lambda c_\lambda$ for some $n_\lambda\in \mathbb C$. This scalar is $n_\lambda = \frac{d!}{\dim V_\lambda}$.
\end{enumerate}

\end{lemma}


\begin{proof}
We can consider $\mathbb C[G] c_\lambda$ as a submodule of $A_{\lambda^*} = \text{Ind}_Q^G (\text{sgn})$. 
The module $M_\lambda = \text{Ind}_P^G (1)$ is generated by the characteristic function of $P1$, and its image in $A_{\lambda^*}$ under $c_\lambda$, understood as a kernel function as in the proof of Proposition \ref{proposition-intertwiners}, is $1c_\lambda \in \mathbb C[G] c_\lambda\subset A_{\lambda^*}$. Equivalently, we can realize $V_\lambda$ as the image of $A_{\lambda^*}$ in $M_\lambda$ under the adjoint operator (given by the same kernel), and then we obtain the submodule $\mathbb C[G] c_\lambda^*\subset M_\lambda$.

The actions of $a_\lambda$, $b_\lambda$ on $V^{\otimes^d}$ are easy to describe from the definitions.

To prove $a_\lambda \cdot x \cdot b_\nu =0$, it is enough to consider the basis elements $x=g\in S_d$, and then by renaming the elements it is enough to consider $g=1$. If $\lambda>\nu$ (lexicographically), there are two elements $k, l$ which belong to the same row in the Young diagram for $\lambda$ and in the same column for $\nu$. If $t=(k,l)$ then $a_\lambda\cdot t = a_\lambda$, $t\cdot b_\nu = -b_\nu$, hence $a_\lambda b_\nu = a_\lambda t \cdot tb_\nu = -a_\lambda b_\nu$, hence is zero. 

The uniqueness (up to scalar) of $c_\lambda$ with this property is a reformulation of Proposition \ref{proposition-intertwiners}, considering such elements, as in the proof of that proposition, as $G^{\text{diag}}$-invariant kernels. 

Clearly, $c_\lambda x c_\lambda$ has this property, therefore is a multiple of $c_\lambda$. To determine the scalar $n_\lambda$, consider the operator of right multiplication by $c_\lambda$, as an endomorphism of $\mathbb C[S_d]$. It acts by $n_\lambda$ on its image $\mathbb C[S_d]c_\lambda$, which is isomorphic to $V_\lambda$, hence its trace is $n_\lambda \dim[V_\lambda]$. On the other hand, the coefficient of the identity element in $c_\lambda$ is $1$, so its trace is equal to $\dim \mathbb C[S_d] = d!$. 
\end{proof}





\begin{definition}
\label{definition-Young-symmetrizer}
The element $c_\lambda$ of the group algebra $\mathbb C[S_d]$ defined above is called the {\it Young symmetrizer} attached to the partition $\lambda$.
\end{definition}



\begin{theorem}
 \label{theorem-Schur-Weyl}
Let $V$ be the standard representation of $\mathfrak{sl}_n$, and consider $V^{\otimes^d}$ as a representation of $S_d\times \mathfrak{sl}_n$. Denote by $\Pi_n$, $\Pi_d'$ the sets of isomorphism classes of irreducible representations of $\mathfrak{sl}_n$, resp.\ $S_d$. There is a map $\sigma:\Pi_d'\to \Pi_n \cup \{0\},$
where $0$ denotes the zero-dimensional representation,
such that 
$$V^{\otimes^d} = \bigoplus_{\pi\in \Pi_d'} \pi \otimes \sigma(\pi)$$
as $S_d\times \mathfrak{sl}_n$-modules.  

The map $\sigma$ is induced by the \emph{Schur functor}, sending a vector space $V$ to $c_\lambda V^{\otimes^d}$ (or, equivalently, to $c_\lambda^* V^{\otimes^d}$), where $c_\lambda$ is the Young symmetrizer of Definition \ref{definition-Young-symmetrizer} (and $c_\lambda^*=b_\lambda a_\lambda$ its adjoint). If $\lambda:\lambda_1 \ge \dots \ge \lambda_m$ is a partition, and $\lambda^*: \lambda_1^* \ge \dots \ge \lambda_r^*$ its dual, the space $c_\lambda V^{\otimes^d}$ is the image of the subspace
$$ \bigwedge^{\lambda_1^*}V \otimes \cdots \otimes \bigwedge^{\lambda_m^*} V$$
of $V^{\otimes^d}$ under the symmetrization map:
$$ V^{\otimes^d} \to S^{\lambda_1} V \otimes \dots \otimes S^{\lambda_m} V$$
(and, respectively, the space $c_\lambda^* V^{\otimes^d}$ is the image of the above product of symmetric powers in the above produce of alternating powers, under the antisymmetrization map). Here, we think of the factors of $V$ as labelled by the boxes of a Young diagram of shape $\lambda$, with symmetric powers taken among the factors in the same row, and exterior powers taken among the factors in the same column.

The map $\sigma$ takes the irreducible representation of $S_d$ parametrized by the partition $\lambda$ to the irreducible $\mathfrak{sl}_n$-module of heighest weight $\lambda_1 \ge \dots \ge \lambda_m \ge 0\ge \dots \ge 0$, if $m \le n$, or to zero, otherwise. 

Consequently, the map $\sigma$ is injective away from the fiber of zero, does not have zero in the image if $n\ge d$, and the resulting map $\sqcup_d \Pi_d' \to \Pi_n \cup\{0\}$ is surjective.
\end{theorem}

\begin{proof}
 The existence of the map $\sigma$ follows from the double centralizer theorem \ref{theorem-double-centralizer} and Theorem \ref{theorem-Schur-Weyl-centralizer}. Under this theorem, $\sigma(V_\lambda) = \Hom_{S_d} (V_\lambda, V^{\otimes^d})$. Realizing $V_\lambda$ as $\mathbb C[S_d]c_\lambda$, by the idempotence of $c_\lambda$ the space $\Hom_{S_d} (V_\lambda, V^{\otimes^d})$ can be identified with $c_\lambda V^{\otimes^d}$, by the map that assigns to a morphism the image of $1c_\lambda$. Equivalently, we can realize $V_\lambda$ as $\mathbb C[S_d]c_\lambda$, and the same argument holds. 
 
 The description of $c_\lambda V^{\otimes^d}$, $c_\lambda^* V^{\otimes^d}$ follows from Lemma \ref{lemma-projectors}. We determine the highest weight, using the realization $\sigma V_\lambda = c_\lambda^* V^{\otimes^d}$. Let $x_1, \dots, x_n$ be a basis for $V$, and consider the vector
 $$ \otimes^{\lambda_1} x_1 \otimes \otimes^{\lambda_2} x_2 \cdots \otimes \otimes^{\lambda_n} x_n \in S^{\lambda_1} V \otimes \dots \otimes S^{\lambda_m} V.$$
 Its image in $\bigwedge^{\lambda_1^*}V \otimes \cdots \otimes \bigwedge^{\lambda_m^*} V$ (where, recall, we are labeling the factors of $V$ according to the boxes in a Young diagram, and antisymmetrize along the columns) is the vector 
 $$ \mathcal A(x_1\otimes x_2 \otimes\cdots\otimes x_{\lambda^*_1}) \otimes \cdots \otimes \mathcal A(x_1\otimes x_2 \otimes\cdots\otimes x_{\lambda^*_r}),$$
 where $\mathcal A$ denotes the antisymmetrization map. 
 
 Evidently, this is an eigenvector for the parabolic subgroup of $\text{GL}_n$ (and, a fortiori, of $\text{SL}_n$) that stabilizes the flag 
 $$ \text{span}(x_1,x_2, x_{\lambda^*_1}) \supset \text{span}(x_1,x_2, x_{\lambda^*_2}) \supset \dots \supset \text{span}(x_1,x_2, x_{\lambda^*_r}),$$
 and its weight is $(\lambda_1, \lambda_2 ,\dots, \lambda_n)$. 
 
 
\end{proof}

\section{The Kazhdan--Lusztig conjectures}
\label{section-Kazhdan-Lusztig}

[TBA]

\input{chapters}


\bibliography{my}
\bibliographystyle{amsalpha}

\end{document}
